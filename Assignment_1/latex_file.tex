\documentclass[journal,12pt,twocolumn]{IEEEtran}

\usepackage{setspace}
\usepackage{gensymb}
\singlespacing
\usepackage[cmex10]{amsmath}

\usepackage{amsthm}
\usepackage{karnaugh-map}
\usepackage{mathrsfs}
\usepackage{txfonts}
\usepackage{stfloats}
\usepackage{bm}
\usepackage{cite}
\usepackage{cases}
\usepackage{subfig}
\usepackage{float}
\usepackage{longtable}
\usepackage{multirow}

\usepackage{enumitem}
\usepackage{mathtools}
\usepackage{steinmetz}
\usepackage{tikz}
\usepackage{circuitikz}
\usepackage{verbatim}
\usepackage{tfrupee}
\usepackage[breaklinks=true]{hyperref}
\usepackage{graphicx}
\usepackage{tkz-euclide}

\usetikzlibrary{calc,math}
\usepackage{listings}
    \usepackage{color}                                            %%
    \usepackage{array}                                            %%
    \usepackage{longtable}                                        %%
    \usepackage{calc}                                             %%
    \usepackage{multirow}                                         %%
    \usepackage{hhline}                                           %%
    \usepackage{ifthen}                                           %%
    \usepackage{lscape}     
\usepackage{multicol}
\usepackage{chngcntr}

\DeclareMathOperator*{\Res}{Res}

\renewcommand\thesection{\arabic{section}}
\renewcommand\thesubsection{\thesection.\arabic{subsection}}
\renewcommand\thesubsubsection{\thesubsection.\arabic{subsubsection}}

\renewcommand\thesectiondis{\arabic{section}}
\renewcommand\thesubsectiondis{\thesectiondis.\arabic{subsection}}
\renewcommand\thesubsubsectiondis{\thesubsectiondis.\arabic{subsubsection}}


\hyphenation{op-tical net-works semi-conduc-tor}
\def\inputGnumericTable{}                                 %%

\lstset{
%language=C,
frame=single, 
breaklines=true,
columns=fullflexible
}
\begin{document}


\newtheorem{theorem}{Theorem}[section]
\newtheorem{problem}{Problem}
\newtheorem{proposition}{Proposition}[section]
\newtheorem{lemma}{Lemma}[section]
\newtheorem{corollary}[theorem]{Corollary}
\newtheorem{example}{Example}[section]
\newtheorem{definition}[problem]{Definition}

\newcommand{\BEQA}{\begin{eqnarray}}
\newcommand{\EEQA}{\end{eqnarray}}
\newcommand{\define}{\stackrel{\triangle}{=}}
\bibliographystyle{IEEEtran}
\raggedbottom
\providecommand{\mbf}{\mathbf}
\providecommand{\pr}[1]{\ensuremath{\Pr\left(#1\right)}}
\providecommand{\qfunc}[1]{\ensuremath{Q\left(#1\right)}}
\providecommand{\sbrak}[1]{\ensuremath{{}\left[#1\right]}}
\providecommand{\lsbrak}[1]{\ensuremath{{}\left[#1\right.}}
\providecommand{\rsbrak}[1]{\ensuremath{{}\left.#1\right]}}
\providecommand{\brak}[1]{\ensuremath{\left(#1\right)}}
\providecommand{\lbrak}[1]{\ensuremath{\left(#1\right.}}
\providecommand{\rbrak}[1]{\ensuremath{\left.#1\right)}}
\providecommand{\cbrak}[1]{\ensuremath{\left\{#1\right\}}}
\providecommand{\lcbrak}[1]{\ensuremath{\left\{#1\right.}}
\providecommand{\rcbrak}[1]{\ensuremath{\left.#1\right\}}}
\theoremstyle{remark}
\newtheorem{rem}{Remark}
\newcommand{\sgn}{\mathop{\mathrm{sgn}}}
\providecommand{\abs}[1]{\left\vert#1\right\vert}
\providecommand{\res}[1]{\Res\displaylimits_{#1}} 
\providecommand{\norm}[1]{\left\lVert#1\right\rVert}
%\providecommand{\norm}[1]{\lVert#1\rVert}
\providecommand{\mtx}[1]{\mathbf{#1}}
\providecommand{\mean}[1]{E\left[ #1 \right]}
\providecommand{\fourier}{\overset{\mathcal{F}}{ \rightleftharpoons}}
%\providecommand{\hilbert}{\overset{\mathcal{H}}{ \rightleftharpoons}}
\providecommand{\system}{\overset{\mathcal{H}}{ \longleftrightarrow}}
	%\newcommand{\solution}[2]{\textbf{Solution:}{#1}}
\newcommand{\solution}{\noindent \textbf{Solution: }}
\newcommand{\cosec}{\,\text{cosec}\,}
\providecommand{\dec}[2]{\ensuremath{\overset{#1}{\underset{#2}{\gtrless}}}}
\newcommand{\myvec}[1]{\ensuremath{\begin{pmatrix}#1\end{pmatrix}}}
\newcommand{\mydet}[1]{\ensuremath{\begin{vmatrix}#1\end{vmatrix}}}
\numberwithin{equation}{subsection}
\makeatletter
\@addtoreset{figure}{problem}
\makeatother
\let\StandardTheFigure\thefigure
\let\vec\mathbf
\renewcommand{\thefigure}{\theproblem}
\def\putbox#1#2#3{\makebox[0in][l]{\makebox[#1][l]{}\raisebox{\baselineskip}[0in][0in]{\raisebox{#2}[0in][0in]{#3}}}}
     \def\rightbox#1{\makebox[0in][r]{#1}}
     \def\centbox#1{\makebox[0in]{#1}}
     \def\topbox#1{\raisebox{-\baselineskip}[0in][0in]{#1}}
     \def\midbox#1{\raisebox{-0.5\baselineskip}[0in][0in]{#1}}
\vspace{3cm}
\title{FPGA LAB \\ Assignment 1}
\author{Yenigalla Samyuktha \\ EE20MTECH14019}
\maketitle
\newpage
\bigskip
\renewcommand{\thefigure}{\theenumi}
\renewcommand{\thetable}{\theenumi}
\noindent Download all codes from 
\begin{lstlisting}
https://github.com/EE20MTECH14019/FPGA-14019/tree/main/Assignment_1
\end{lstlisting}
%
and latex-tikz codes from 
%
\begin{lstlisting}
https://github.com/EE20MTECH14019/FPGA-14019/tree/main/Assignment_1
\end{lstlisting}
\section{Problem}
\noindent(CBSE/CS/2015-1/6.c) Derive a Canonical POS expression for a Boolean function $F$, represented by the following truth table \ref{tt}:
\begin{table}[h!]
\centering
\begin{tabular}{|l|l|l|c|} 
\hline
$P$ & $Q$ & $R$ & $F(P,Q,R)$  \\ 
\hline
0 & 0 & 0 & 1         \\ 
\hline
0 & 0 & 1 & 0         \\ 
\hline
0 & 1 & 0 & 0         \\ 
\hline
0 & 1 & 1 & 1         \\ 
\hline
1 & 0 & 0 & 1         \\ 
\hline
1 & 0 & 1 & 0         \\ 
\hline
1 & 1 & 0 & 0         \\ 
\hline
1 & 1 & 1 & 1         \\
\hline
\end{tabular}
\caption{Truth table for Function F}
\label{tt}
\vspace{-8mm}
\end{table}
\section{Solution}
\noindent From the truth table \ref{tt}, the function $F(P,Q,R)$ can be represented in the Canonical POS form as follows:\\
As $F$ is logic 0 for four input combinations, the corresponding max terms are, 
\begin{align}
    M_1 = (P+Q+\bar{R})\\
    M_2 = (P+\bar{Q}+R)\\
    M_5 = (\bar{P}+Q+\bar{R})\\
    M_6 = (\bar{P}+\bar{Q}+R)
\end{align}
And the Canonical POS expression can be expressed using these max terms as.
\begin{align}
F = M_1 \cdot M_2 \cdot M_5 \cdot M_6 \\
F = \prod M\left ( 1,2,5,6 \right )\\
F = (P+Q+\bar{R})\cdot(P+\bar{Q}+R)\cdot\nonumber\\
(\bar{P}+Q+\bar{R})\cdot(\bar{P}+\bar{Q}+R)\label{FPOS}
\end{align}
\underline{Implementation using two input NAND gates}\\\\
Minimizing the function \eqref{FPOS} using K-maps,
\begin{figure}[H]
\centering
 \begin{karnaugh-map}[4][2][1][$Q R$][$P$]
        \minterms{0,3,4,7}
        \maxterms{1,2,5,6}
        \implicant{1}{5}
        \implicant{2}{6}
        
    \end{karnaugh-map}
    \vspace{-6mm}
\end{figure}
\noindent From the K-Map, the minimized POS form is,
\begin{align}
F = (Q+\bar{R})\cdot(\bar{Q}+R)\\
F = \bar{Q}\cdot\bar{R}+Q\cdot R
\end{align}
Using Demorgan's law,
\begin{align}
F = \overline{\overline{(\bar{Q}\cdot\bar{R})}\cdot\overline{(Q\cdot R)}}\label{Fnand}
\end{align}
Now, implementing the Boolean function $F$ in \eqref{Fnand} using two input NAND gates:
\begin{figure}[H]
\centering
\resizebox{\columnwidth}{!}
    {
    \begin{circuitikz}
    \node (Q) at (0,4.5) {$Q$};
    \node (R) at (1,2.7) {$R$};
    \node (F) at (10.8,2.5) {$F = \overline{\overline{(\bar{Q}\cdot\bar{R})}\cdot\overline{(Q\cdot R)}}$};
    \node[nand port] at ($(Q) + (2.5, -0.2)$) (nand1) {};
    \node[nand port] at ($(Q) + (3.5, -1.8)$) (nand2) {};
    \node[nand port] at ($(Q) + (5.5, -1.0)$) (nand3) {};
     \node[nand port] at ($(Q) + (4.5, -3.6)$) (nand4) {};
     \node[nand port] at ($(Q) + (8.5, -2.0)$) (nand5) {};
     
    \draw 
    (nand1.in 1) |- (nand1.in 2)
  (nand2.in 1) |- (nand2.in 2)
  (nand1.out) |- (nand3.in 1)
  (nand2.out) |- (nand3.in 2)
  (nand3.out) |- (nand5.in 1)
  (nand4.out) |- (nand5.in 2)
  (0.3,4.3)  |- (1.1,4.3) %Q to nand1
  (1.3,2.7)  |- (2.1,2.7) %R to nand2
  (1.5,2.7)  |- (nand4.in 1) %R to nand4
  (0.5,4.3)  |- (nand4.in 2); %Q to nand4
  \filldraw 
(1.5,2.7) circle (2pt) node[align=left,   below] {}
(0.5,4.3) circle (2pt) node[align=left,   below] {}
(1.1,4.3) circle (2pt) node[align=left,   below] {}
(2.1,2.7) circle (2pt) node[align=left,   below] {};
        
\end{circuitikz}

    }
\caption{Implementation using two input NAND gates}
\end{figure}
\end{document}
